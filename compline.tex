\documentclass[a5paper,12pt,twoside,openany,oldfontcommands]{memoir}
\usepackage[british,latin]{babel}
\usepackage{fontspec}
\defaultfontfeatures{Ligatures=TeX}
\setmainfont[Numbers=OldStyle]{Linux Libertine O}
%\setmainfont[BoldFont={Mezalia Extra Bold},
%             ItalicFont={Mezalia Cursive}]{Mezalia Regular}
\newfontfamily{\anavio}{Linux Libertine O}
%\newfontfamily{\vectis}{Mezalia Initials}
\newfontfamily{\vectis}{Kingthings Versalis}
%\newfontface{\vectis}{Vectis Bold}
\newfontface{\versicles}{Versiculum}
%\newfontface{\inglesh}{Mezalia Sans}
\newfontface{\inglesh}{Linux Biolinum O}
\usepackage{parallel}
%\usepackage{lettrine}
\newfontface{\cruci}{Juuji}
\usepackage[autocompile]{gregoriotex}
%\usepackage{canoniclayout}
%\settrimmedsize{9in}{6in}{*}
\setlength{\trimtop}{0pt}
\setlength{\trimedge}{0pt}
%\addtolength{\trimedge}{-\paperwidth}
\settypeblocksize{170mm}{110mm}{*} % Was 101mm in a5
\setulmargins{*}{*}{1.4}
\setlrmargins{*}{*}{1.2}
\setheadfoot{\onelineskip}{2\onelineskip}
\setheaderspaces{*}{.8\onelineskip}{*}
\checkandfixthelayout

\tolerance=1000
\pretolerance=800

\setlength{\parskip}{6pt plus 2pt minus 1pt}
\setlength{\parindent}{0pt}
\setlength{\stanzaskip}{6pt plus 2pt minus 1pt}

\newlength{\gcolwidth}
\setlength{\gcolwidth}{0.48\textwidth}
\newlength{\lcolwidth}
\setlength{\lcolwidth}{0.53\textwidth}
\newlength{\vcolwidth}
\setlength{\vcolwidth}{0.43\textwidth}

%\renewcommand\gretranslationformat[1]{{\inglesh \fontsize{9}{11}\selectfont #1}}
\grechangestyle{translation}{\inglesh \fontsize{9}{11}\selectfont}
\newcommand\translationfont[1]{{\inglesh \fontsize{9}{11}\selectfont #1}}

\grechangestyle{initial}{\vectis \fontsize{32}{32}\selectfont}

\grechangestaffsize{16}


%%%%%%%%%%%%%%%%%%%
%
% Parallel columns
%
%%%%%%%%%%%%%%%%%%%

%\def\latin#1{%
%\selectlanguage{latin}
%\ParallelLText{\ifnum \value{versecount}>0 \stepcounter{versecount} \theversecount .~\fi\ifnum \value{versecount}=0 \strut\fi\selectlanguage{latin}#1\selectlanguage{british}\strut}%
%\selectlanguage{british}%
%\relax %
%}

%\def\vern#1{%
%\ParallelRText{\translationfont{#1}\strut}%
%%\ParallelLText{\ \vspace*{1mm}}\ParallelRText{\ \vspace*{1mm}}%
%\ParallelPar %
%%\kern -1mm%
%\relax %
%}
%\def\startParallel{\begin{Parallel}{\lcolwidth}{\vcolwidth}}
%\def\stopParallel{\end{Parallel}}

\newcounter{versecount}
\setcounter{versecount}{0}


%%%%%%%%%%%%%%%%%%%%%%%%%%%%
%
% Interlinear translation code:
%
%%%%%%%%%%%%%%%%%%%%%%%%%%%%

\def\latin#1{\ifnum \value{versecount}>0 \stepcounter{versecount} \theversecount .~\fi\selectlanguage{latin}#1\selectlanguage{british}\strut\par}
\def\vern#1{\translationfont{#1\par}\par\pagebreak[3]}


\def\startParallel{}
\def\stopParallel{}

\setlength{\parskip}{3pt plus 2pt minus 1pt}

\def\firstlatin#1#2#3{%
\latin{%
\noindent\begin{minipage}[t]{\ParallelLWidth}%
\lettrine{#1}{#2}#3\strut%\vspace{0.9mm}\vspace{0.13cm}%
\end{minipage}%
}%
%\vspace{-0.2cm}%
}

\def\firstvern#1#2#3{%
\vern{%
\noindent\begin{minipage}[t]{\ParallelRWidth}%
\lettrine{#1}{#2}\translationfont{#3}\strut%\vspace{0.9mm}\vspace{0.13cm}%
\end{minipage}%
}%\vspace{-0.2cm}%
}

\newcommand\maltese{{\tiny \cruci V}\ }

\renewcommand\Vbar{\makebox[1em][l]{\versicles v}}
\renewcommand\Rbar{\makebox[1em][l]{\versicles r}}



\newcommand\rubrics[1]{\emph{#1}}
\newcommand\bibleref[2]{\smallskip\par\pagebreak[3]\textbf{#1} \hfill \textit{#2}}
\newcommand\psalm[1]{\bigskip\relax\par\pagebreak[3] ~ \hfill \textbf{Psalm #1.} \hfill ~}

\title{Ad Completorium}

\chapterstyle{dash}
\nouppercaseheads

\begin{document}


\setcounter{secnumdepth}{0}

\maketitle

\tableofcontents*


\chapter{The beginning}
\label{beginning}

\rubrics{All stand. The lector stands in the middle, before the altar.  He turns to the tabernacle, bows, and sings:}

\gresetinitiallines0
\gregorioscore{common/jubedomne}

\rubrics{[If the celebrant is a priest, the lector sings \emph{Jube, domne benedicere} and bows to the priest, asking his blessing.]}

\rubrics{The celebrant intones the blessing:}

\gregorioscore{common/noctem}

\eject

\rubrics{The lector rises, and, facing the altar, sings the short lesson, genuflacting at the words \emph{Tu autem\ldots }:}

\bibleref{Short Lesson.}{1. Peter 5.}

\gresetinitiallines1
\gregorioscore{common/complineread}

\rubrics{All make the sign of the cross as the celebrant sings:}

\gresetinitiallines0
\gregorioscore{common/adjutorium}

\gregorioscore{common/quifecit}

\rubrics{\emph{Pater noster.} altogether in silence.}

\rubrics{Then, if the celebrant is a priest, he says the \emph{Confiteor} entirely recto tono, in a low voice.  The Choir replies with \emph{Misereátur tui.}}

\rubrics{If there is no priest present all recite the \emph{Confiteor} as follows:}

\startParallel
\latin{Confiteor Deo omnipoténti, %\parfillskip0pt 
beát\ae\ Marí\ae\ semper Vírgini, beáto
Michaéli Archángelo, beáto Joánni Baptíst\ae, sanctis Apóstolis Petro
et Paulo, et ómnibus Sanctis, [\textit{(if a priest presides)} et tibi, Pater:] quia peccávi nimis cogitatióne,
verbo et ópere: \textit{(strike breast three times)} \textbf{mea
culpa, mea culpa, mea máxima culpa.} Ideo precor beátam Maríam semper
Vírginem, beátum Michaélem Archángelum, beátum Joánnem Baptístam,
sanctos Apóstolos Petrum et Paulum, et omnes Sanctos, [\textit{(if a priest presides)} et te, Pater,] oráre
pro me ad Dóminum Deum nostrum.}
\vern{I confess to Almighty God, to blessed Mary %\parfillskip0pt 
ever Virgin, to blessed Michael the Arch\-angel, to blessed John the
Baptist, to the holy Apostles Peter and Paul, to all the Saints, [and
to you, Father,] that I have sinned exceedingly, in thought, word,
and deed: \textit{(strike breast three times)} through my fault,
through my fault, through my most grievous fault. Therefore I beseech
blessed Mary ever Virgin, blessed Michael the Arch\-angel, blessed John
the Baptist, the holy Apostles Peter and Paul, all the Saints, [and
you, Father,] to pray for me to the Lord our God.}
%\latin{}
%\vern{}
\stopParallel
\smallskip
\rubrics{The celebrant next says:}
\startParallel
\latin{Misereátur nostri omnípotens Deus, et dimíssis peccátis
nostris, perdúcat nos ad vitam \ae térnam. \Rbar Amen.}
\vern{May Almighty God have mercy upon you, forgive you your sins, and
bring you to life everlasting. Amen.}

%\latin{\Rbar Amen.}
%\vern{Amen.}


\latin{Indulgéntiam, \maltese absolutiónem, et remissiónem peccatórum
nostrórum, tríbuat nobis omnípotens et miséricors Dóminus. \Rbar Amen.}
\vern{May the Almighty and merciful Lord grant us pardon, \maltese absolution,
and remission of our sins. Amen.}

%\latin{\Rbar Amen.}
%\vern{Amen.}

\stopParallel

%\eject

\rubrics{The celebrant then sings:}

\gregorioscore{common/converte}

%\bibleref{Ferial Tone.}{}

\gresetinitiallines1
\greannotation{\Vbar}
\gregorioscore{common/deusinadjutorium_ferial}


\chapter{The Psalter}

\section{Sunday and Feasts}
\label{sunday}

\rubrics{The celebrant intones the antiphon, either \emph{Miserere} most of the year or \emph{Alleluia} during Paschaltide.  The cantor may pre-intone the antiphon for the celebrant according to custom.  All take up the antiphon.  The cantors return to the middle, genuflect, then intone the first verse of the psalm to the asterisk.  They bow to the side of the choir which is to continue the verse (usually the celebrant's side), genuflect, bow to one another, and go their places as all in choir sit, cover with the biretta and proceed with the psalm.}

\gresetinitiallines1
%\grechangedim{annotationseparation}{-1pt}{scalable}
%\grechangedim{annotationraise}{3pt}{scalable}
%\greannotation{\textsc{ant.}}
%\greannotation{8 G}
\gregorioscore{Psalms/sundayantiphoninit}
%\grechangedim{annotationraise}{0pt}{scalable}

\rubrics{P.~T.~stands for Paschal Time.  T.~P.~ stands for Tempore Paschalis.}

\greannotation{P. T.}
\gregorioscore{Psalms/alleluiapaschalinit}

\psalm{4}

\gresetinitiallines0
\gregorioscore{Psalms/sundaypsalmsinit1}

\setcounter{versecount}{1}
\startParallel
\input Psalms/sundaypsalms1
\stopParallel

\psalm{90}

\setcounter{versecount}{1}
\gregorioscore{Psalms/sundaypsalmsinit2}

\startParallel
\input Psalms/sundaypsalms2
\stopParallel

\psalm{133}

\gregorioscore{Psalms/sundaypsalmsinit3}

\setcounter{versecount}{1}
\startParallel
\input Psalms/sundaypsalms3
\stopParallel

\gresetinitiallines1
\gregorioscore{Psalms/sundayantiphon}

\greannotation{P. T.}
\gregorioscore{Psalms/alleluiapaschal}


\section{Monday}

\greannotation{Ant.}
\greannotation{8. G}
\gregorioscore{Psalms/mondayinitantiphon}

\greannotation{P. T.}
\greannotation{Ant.}
\gregorioscore{Psalms/alleluiapaschalinit}

\bigskip

\psalm{6}

\gresetinitiallines0
\gregorioscore{Psalms/mondaypsalmsinit1}

\setcounter{versecount}{1}
\startParallel
\input Psalms/mondaypsalms1
\stopParallel

\psalm{7. I}

\gregorioscore{Psalms/mondaypsalmsinit2}

\setcounter{versecount}{1}
\startParallel
\input Psalms/mondaypsalms2
\stopParallel

\psalm{7. II}

\gregorioscore{Psalms/mondaypsalmsinit3}


\setcounter{versecount}{1}
\startParallel
\input Psalms/mondaypsalms3
\stopParallel

\gresetinitiallines1
\greannotation{Ant.}
\greannotation{8. G}
\gregorioscore{Psalms/mondayantiphon}

\greannotation{P. T.}
\greannotation{Ant.}
\gregorioscore{Psalms/alleluiapaschal}

\section{Tuesday}

\greannotation{Ant.}
\greannotation{8. G}
\gregorioscore{Psalms/tuesdayantiphoninit}

\greannotation{P. T.}
\greannotation{Ant.}
\gregorioscore{Psalms/alleluiapaschalinit}

\psalm{11}

\gresetinitiallines0
\gregorioscore{Psalms/tuesdaypsalmincipit1}

\setcounter{versecount}{1}
\startParallel
\input Psalms/tuesdaypsalms1
\stopParallel

\psalm{12}

\gregorioscore{Psalms/tuesdaypsalmincipit2}

\setcounter{versecount}{1}
\startParallel
\input Psalms/tuesdaypsalms2
\stopParallel

\psalm{15}

\gregorioscore{Psalms/tuesdaypsalmincipit3}

\setcounter{versecount}{1}
\startParallel
\input Psalms/tuesdaypsalms3
\stopParallel

\gresetinitiallines1
\greannotation{Ant.}
\greannotation{8. G}
\gregorioscore{Psalms/tuesdayantiphon}

\greannotation{P. T.}
\greannotation{Ant.}
\gregorioscore{Psalms/alleluiapaschal}


\section{Wednesday}

\rubrics{for Paschaltide, see p.~\pageref{wed-pt}.}

\subsection{Outside Paschal Time}


\greannotation{Ant.}
\greannotation{3. a}
\gregorioscore{Psalms/wednesdayantiphoninit}

\psalm{33. I}

\gresetinitiallines0
\gregorioscore{Psalms/Wednesdaypsalminit1}

\setcounter{versecount}{1}
\startParallel
\input Psalms/wednesdaypsalm1
\stopParallel


\psalm{33. II}

\gregorioscore{Psalms/Wednesdaypsalminit2}

\setcounter{versecount}{1}
\startParallel
\input Psalms/wednesdaypsalm2
\stopParallel


\psalm{60}

\gregorioscore{Psalms/Wednesdaypsalminit3}


\setcounter{versecount}{1}
\startParallel
\input Psalms/wednesdaypsalm3
\stopParallel

\gresetinitiallines1
\greannotation{Ant.}
\greannotation{3. a}
\gregorioscore{Psalms/wednesdayantiphon}


\section{Thursday}

\greannotation{Ant.}
\greannotation{8. G}
\gregorioscore{Psalms/thursdayantiphoninit}

\greannotation{P. T.}
\greannotation{Ant.}
\gregorioscore{Psalms/alleluiapaschalinit}

\psalm{69}

\gresetinitiallines0
\gregorioscore{Psalms/thursdaypsalminit1}

\setcounter{versecount}{1}
\startParallel
\input Psalms/thursdaypsalm1
\stopParallel

\psalm{70. I}

\gregorioscore{Psalms/thursdaypsalminit2}

\setcounter{versecount}{1}
\startParallel
\input Psalms/thursdaypsalm2
\stopParallel


\psalm{70. II}

\gregorioscore{Psalms/thursdaypsalminit3}


\setcounter{versecount}{1}
\startParallel
\input Psalms/thursdaypsalm3
\stopParallel


\gresetinitiallines1
\greannotation{Ant.}
\greannotation{8. G}
\gregorioscore{Psalms/thursdayantiphon}

\greannotation{P. T.}
\greannotation{Ant.}
\gregorioscore{Psalms/alleluiapaschal}




\section{Friday}

\rubrics{for Paschaltide, see p.~\pageref{fri-pt}.}

\subsection{Outside Paschal Time.}


\greannotation{Ant.}
\greannotation{7. c}
\gregorioscore{Psalms/fridayantiphoninit}

\psalm{76. I}

\gresetinitiallines0
\gregorioscore{Psalms/fridaypsalminit1}

\setcounter{versecount}{1}
\startParallel
\input Psalms/fridaypsalms1
\stopParallel


\psalm{76. II}

\gregorioscore{Psalms/fridaypsalminit2}

\setcounter{versecount}{1}
\startParallel
\input Psalms/fridaypsalms2
\stopParallel


\psalm{85}

\gregorioscore{Psalms/fridaypsalminit3}

\setcounter{versecount}{1}
\startParallel
\input Psalms/fridaypsalms3
\stopParallel



\gresetinitiallines1
\greannotation{Ant.}
\greannotation{7. c}
\gregorioscore{Psalms/fridayantiphon}



\section{Saturday}

\rubrics{For Paschaltide see p.~\pageref{sat-pt}.}

\subsection{Outside Paschal Time.}


\greannotation{Ant.}
\greannotation{5. a}
\gregorioscore{Psalms/saturdayantiphoninit}

\psalm{87}

\gresetinitiallines0
\gregorioscore{Psalms/saturdaypsalminit1}

\setcounter{versecount}{1}
\startParallel
\input Psalms/saturdaypsalms1
\stopParallel


\psalm{102. I}

\gregorioscore{Psalms/saturdaypsalminit2}

\setcounter{versecount}{1}
\startParallel
\input Psalms/saturdaypsalms2
\stopParallel


\psalm{102. II}

\gregorioscore{Psalms/saturdaypsalminit3}

\setcounter{versecount}{1}
\startParallel
\input Psalms/saturdaypsalms3
\stopParallel


\gresetinitiallines1
\greannotation{Ant.}
\greannotation{5. a}
\gregorioscore{Psalms/saturdayantiphon}

\setcounter{versecount}{0}


\chapter{The Hymn}

\rubrics{The cantors preintone the first line of the hymn which the celebrant then intones aloud.  The celebrant's side sing the first verse, then the other side for the second verse then all together for the doxology.}

\section*{Sundays and minor feasts:}

\greannotation{8.}
\gregorioscore{Hymn/telucisRegular}

\section*{Feasts:}

\greannotation{4.}
\gregorioscore{Hymn/telucisFeasts}

\section*{Ferias:}

\greannotation{8.}
\gregorioscore{Hymn/telucisFeria}

\begin{minipage}[t]{\gcolwidth}
Procul recédant sómnia,\\
Et nóctium phantásmata:\\
Hostémque nostrum cómprime,\\
Ne polluántur córpora.\\
 ~ \\
Praésta, Pater piíssime,\\
Patríque compar Unice,\\
Cum Spíritu Paráclito,\\
Regnans per omne saéculum.\\
 ~ \\
Amen.
\end{minipage}
\begin{minipage}[t]{\gcolwidth}
\translationfont{To thee before the close of day,\\
Creator of the world, we pray\\
That, with thy wonted favour, thou\\
Wouldst be our guard and keeper now.

From all ill dreams defend our sight,\\
From fears and terrors of the night;\\
Withhold from us our ghostly foe,\\
That spot of sin we may not know.

O Father, that we ask be done,\\
Through Jesus Christ, thine only Son,\\
Who, with the Holy Ghost and thee,\\
Doth live and reign eternally. Amen.

}
\end{minipage}

\section{Advent:}

\rubrics{This Tone is used from the Saturday before the first Sunday of Advent to the night before Christmas Eve, even on Feasts. But on the Feast of the Immaculate Conception and during its Octave, the ordinary Tone for the Feasts of the Blessed Virgin Mary is used, except on the Sunday within the Octave, and on the Octave day itself if this should fall on a Sunday, when the Tone for Advent is used, with the doxology of the Sunday.}

\greannotation{2.}
\gregorioscore{Hymn/telucisAdvent}

%\newpage

\section{Christmas:}

\label{TLchristmas}
\rubrics{This Tone (or the following alternative) is used from Christmas Eve to the 4th January, even in the Office of Saints.}

\greannotation{8.}
\gregorioscore{Hymn/telucisChristmas}

\subsection{Alternative tune Jesu Redemptor omnium}

\rubrics{This is the tune for the Christmas Vespers hymn.}

\greannotation{1.}
\gregorioscore{Hymn/telucisChristmasVespers}

\goodbreak

\section{Epiphany:}

\rubrics{From the Eve of the Epiphany on the 5th January to the Octave of the Epiphany and feast of the Baptism of Our Lord on the 13th January.}

\greannotation{8.}
\gregorioscore{Hymn/telucisEpiphany}

\goodbreak


\section{Holy Family:}

\rubrics{For the Sunday in the octave of the Epiphany (7-13 February). The same tune is sung as for the Epiphany, but the last verse (also known as the doxology) has a different text.  Formerly many feasts had proper doxologies.}

%\greannotation{8.}
\gregorioscore{Hymn/telucisHolyFamily}

\goodbreak



\section{Lent:}

\rubrics{This tone is used from Ash Wednesday, or Saturday before the first Sunday of Lent to the Friday before Passion Sunday, at the Office of the Sunday and of the feria only.}

\greannotation{2.}
\gregorioscore{Hymn/telucisLent}

\goodbreak


\section{Passiontide:}

\rubrics{This tone is used from the Saturday before Passion Sunday to Wednesday of Holy Week, even on Feats, unless the contrary be indicated.}

\greannotation{2.}
\gregorioscore{Hymn/telucisPassiontide}

\goodbreak


\section{Easter:}

\rubrics{The first week of Easter has its own thing, see the Appendix on page \pageref{EasterOctave}}

\rubrics{The following tone is used from Saturday in Albis (the first Saturday after Easter) to the Tuesday before Ascension.}

\greannotation{8.}
\gregorioscore{Hymn/telucisEaster}

\goodbreak


\section{Ascension:}

\rubrics{From the night before the Ascension to the Friday before Pentecost.}

\greannotation{4.}
\gregorioscore{Hymn/telucisAscension}

\goodbreak


\section{Pentecost:}

\rubrics{From the night before Pentecost to the Friday before Trinity Sunday.}

\greannotation{1.}
\gregorioscore{Hymn/telucisPentecost}

%\section{Corpus Christi:}

%\rubrics{}
% 1946 book says to use Christmas tune with "Jesu tibi sit gloria, Qui natus es de Virgine"
% for the whole octave
% it's in Latin though so :
% Ad completorium et Horas, Hymni cantantur in tono posito ad Horas
% in Nativitate Domini, 367, et in fine dicitur Jesu tibi sit gloria, Qui natus es de
% Virgine, per totam Octavam.  Eodem modo per octavam terminatur omnes Hymni ejusdem metri, etiam in Officio Sanctorum, nisi aliter notetur.

%\gregorioscore{Hymn/telucisCorpusChristi}

\goodbreak

\section{Corpus Christi}

\rubrics{The 1918 Antiphonale and 1946 Liber say to use the tone for Christmas (page \pageref{TLchristmas}) for the feast of Corpus Christi and through the octave}

\section{Sacred Heart:}

\rubrics{Sung from the eve of the feast and throughout the Octave.}

\greannotation{3.}
\gregorioscore{Hymn/telucisSacredHeart}

\goodbreak


\section{Christ the King:}

\rubrics{This one is the same tune as for Pentecost except with a different doxology. Sung for the Saturday and Sunday.}

\greannotation{1.}
\gregorioscore{Hymn/telucisChristusRex}

\section{Blessed Virgin Mary}

\greannotation{2.}
\gregorioscore{Hymn/telucis_bvm}


\chapter{Little Chapter and Responsory to the end}

\rubrics{The celebrant chants the Little Chapter:}

\gregorioscore{common/tuautem}



\textbf{Short Responsory during the year.}

\rubrics{The cantors lead the Short Responsory and the Versicle.}

\gregorioscore{common/inmanus_duringyear}

\gresetinitiallines0
\gregorioscore{common/custodi}

\rubrics{During Passiontide the \emph{Glória Patri} in the short responsory is omitted.}

\bigskip

\textbf{Short Responsory during Advent.}

\gresetinitiallines1
\gregorioscore{common/inmanus_advent}

\gresetinitiallines0
\gregorioscore{common/custodi_advent}


\bigskip

\textbf{Short Responsory during Easter.}

\gresetinitiallines1
\gregorioscore{common/inmanus_paschal}

\gresetinitiallines0
\gregorioscore{common/custodi_paschal}


%\chapter{The Canticle and Collect}
\section{The Canticle}

\rubrics{The celebrant intones the \emph{Salve nos}.  The cantors intone the \emph{Nunc dimittis} and bow to the celebrant's side to continue it, genuflect and return to their places}

\gresetinitiallines1
\greannotation{Ant.}
\greannotation{3. a.}
\gregorioscore{common/salvanosinit}

\label{canticle}
\bibleref{Canticle of Simeon.}{Luke 2.}

\gresetinitiallines0
\gregorioscore{common/nuncdimittis}

\gresetinitiallines1
\greannotation{Ant.}
\gregorioscore{common/salvanos}

\section{The Collect}
\label{collect}

\rubrics{If the celebrant is a priest he sings \emph{Dominus vobiscum.} and the choir replies \emph{Et cum spiritu tuo.} Otherwise the celebrant sings \emph{Domine exaudi} as below.}

\startParallel
\latin{\Vbar Dómine, exáudi oratiónem meam.}
\vern{O Lord, hear my prayer.}
\latin{\Rbar Et clamor meus ad te véniat.}
\vern{And let my cry come unto Thee.}
\latin{Orémus.}
\vern{Let us pray.}
\latin{Vísita, quæsumus, Dómine, habitatiónem istam, et omnes insídias inimíci ab ea lónge repélle: \dag\ Ángeli tui sancti hábitent in ea, qui nos in pace custódiant; * et benedíctio tua sit super nos semper.
Per Dóminum nostrum Jesum Christum, Filium tuum: \dag\ qui tecum vivit et regnat in unitáte Spíritus Sancti Deus, * per ómnia saécula sæculórum. }
\vern{Visit we beseech thee, O Lord, this dwelling, and drive far from it the snares of the enemy. Let thy holy angels dwell herein to preserve us in peace and let thy blessing be always upon us.
Through Jesus Christ, thy Son our Lord, Who liveth and reigneth with thee, in the unity of the Holy Ghost, ever one God, world without end.}

\latin{\Rbar Amen.}
\vern{Amen.}
\latin{\Vbar Dómine, exáudi oratiónem meam.}
\vern{O Lord, hear my prayer.}
\latin{\Rbar Et clamor meus ad te véniat.}
\vern{And let my cry come unto Thee.}

\rubrics{The cantors chant:}

\latin{\Vbar Benedicámus Dómino.}
\vern{Let us bless the Lord.}
\latin{\Rbar Deo grátias.}
\vern{Thanks be to God.}

\rubrics{The celebrant gives the blessing:}

\latin{Benedícat et custódiat nos omnípotens et miséricors Dóminus, \maltese Pater, et Fílius, et Spíritus Sanctus.}
\vern{The almighty and merciful Lord the Father, the Son and the Holy Spirit bless us and keep us}

\latin{\Rbar Amen.}
\vern{Amen.}
\stopParallel

\rubrics{Then the celebrant intones the Marian Antiphon of the season: \emph{Alma Redemptoris Mater} for Advent and Christmas through to 1st February, p.~\pageref{almaredem}; \emph{Ave Regina Caelorum} for the Purification of Our Lady 2nd February to Holy Wednesday, p.~\pageref{averegina}; \emph{Regina Caeli} for Easter, p.~\pageref{reginacaeli} and lastly the \emph{Salve Regina}, p.~\pageref{salveregina}.}

\rubrics{Outside of Eastertide, on Monday--Friday, without exception even for greater feasts, all, except the Officiant, now kneel for the Final Marian Antiphon. On Saturday and Sunday nights, all continue standing. During Eastertide, all stand every night for the \emph{Regina coeli}. The Officiant stands and intones the applicable Marian Antiphon according to the Season, after which, he also kneels if it be Monday--Friday outside of Eastertide. All continue singing the Marian Antiphon straight through with no alternations.}



Each Marian Antiphon has a Solemn Tone and a Simple Tone. The Solemn Tone is sung on all Saturday and Sunday nights, on all First and Second Class Feasts (when the Office is of the Feast), and on each night during the Octaves of Christmas and Pentecost. The Simple Tone is sung on all other days inclusive of all Ferial Days of any rank (e.g. Ash Wednesday, weekdays of Holy Week).

After the Marian antiphon is finished, the Cantor says the applicable versicle and all make the corresponding the response. Then the Officiant alone stands and says the prayer to which all respond \emph{Amen}.

All making the Sign of the Cross, the Officiant says in low, recto tono voice, the \emph{Divinum auxilium\ldots} and respond \emph{Amen.} Thus ends Compline according to the normal arrangement of the 1962 Breviarium Romanum.

\chapter{The Marian Antiphon}

\section{Advent and Christmas}
\label{almaredem}
\bibleref{Solemn tone:}{}

\greannotation{Ant.}
\greannotation{5.}
\gregorioscore{MarianAntiphons/almaredemptoris_solemn}

\bibleref{Simple tone:}{}

\greannotation{5.}
\gregorioscore{MarianAntiphons/almaredemptoris}

\begin{verse}
\translationfont{Loving Mother of our Saviour, hear thou thy people’s cry\\
Star of the deep and Portal of the sky!\\
Mother of Him who thee made from nothing made.\\
Sinking we strive and call to thee for aid:\\
Oh, by what joy which Gabriel brought to thee,\\
Thou Virgin first and last, let us thy mercy see.

}
\end{verse}

\rubrics{In Advent :}

\startParallel
\latin{\Vbar Angelus D\'omini nunti\'avit Mar\'\i \ae .}
\vern{The angel of the Lord declared unto Mary.}
\latin{\Rbar Et conc\'epit de Sp\'\i ritu Sancto.}
\vern{And she conceived by the Holy Spirit.}
\latin{Or\'emus.}
\vern{Let us pray.}
\latin{Gr\'atiam tuam, qu\'\ae sumus D\'omine, m\'entibus nostris inf\'unde : \dag \ ut qui, Angelo nunti\'ante, Christi F\'\i lii tui Incarnati\'onem cogn\'ovimus, * per passi\'onem ejus et crucem ad resurrecti\'onis gl\'oriam perduc\'amur.  Per \'eumdem Christum D\'ominum nostrum. \Rbar Amen.}
\vern{Pour forth, we beseech thee, O Lord, thy grace into our hearts, that we to whom the incarnation of Christ thy Son was made known by the message of an angel, may by his Passion and Cross be brought to the glory of his resurrection.  Through the same Christ our Lord. Amen.}
\latin{\Vbar Divínum auxílium máneat semper nobíscum. \Rbar Amen.}
\vern{May the divine assistance remain always with us. Amen.}
\stopParallel


\bigskip

\rubrics{From the 24th December to 1st February :}

\startParallel
\latin{\Vbar Post partum Virgo inviol\'ata permans\'\i sti.}
\vern{After childbirth, O Virgin, thou didst remain inviolate.}
\latin{\Rbar Dei G\'enitrix, interc\'ede pro nobis.}
\vern{O Mother of God, plead for us.}
\latin{Or\'emus.}
\vern{Let us pray.}
\latin{Deus, qui salutis \ae t\'ern\ae , be\'atae \ Mar\'\i \ae \ virginit\'ate fec\'unda, hum\'ano g\'eneri pr\'\ae mia pr\ae stit\'\i sti : \dag \ tr\'\i bue, qu\'\ae sumus; ut ipsam pro nobis interc\'edere senti\'amus, * per quam mer\'uimus auct\'orem vit\ae \ susc\'\i pere, D\'ominum nostrum Jesum Christum F\'\i lium tuum. \Rbar Amen}
\vern{O God, Who by the fruitful virginity of blessed Mary, hast given to mankind the rewards of eternal salvation: grant, we beseech You, that we may experience her intercession for us, by whom we deserved to receive the Author of life, our Lord Jesus Christ, Your Son. Amen.}
\latin{\Vbar Divínum auxílium máneat semper nobíscum. \Rbar Amen.}
\vern{May the divine assistance remain always with us. Amen.}
\stopParallel

%\startParallel
%\latin{\Vbar Divínum auxílium máneat semper nobíscum.}
%%\vern{May the divine assistance remain always with us.}
%\latin{\Rbar Amen.}
%\vern{Amen.}
%\stopParallel

\section{Lent}
\label{averegina}

\rubrics{From 2nd February to Holy Wednesday.}

\bibleref{Solemn tone:}{}

\greannotation{Ant.}
\greannotation{6.}
\gregorioscore{MarianAntiphons/averegina_solemn}

\bibleref{Simple tone:}{}

\greannotation{6.}
\gregorioscore{MarianAntiphons/averegina}

\begin{verse}
\translationfont{
Hail, Queen of heaven; Hail, Mistress of the Angels;\\
Hail, root of Jesse;
Hail, the gate through which the Light rose over the earth.\\
Rejoice, Virgin most renowned and of unsurpassed beauty.

Farewell, O most beautiful one,
and pray for us to Christ.

}
\end{verse}

\startParallel
\latin{\Vbar Dign\'are me laud\'are te, Virgo sacr\'ata.}
\vern{Vouchsafe that I may praise thee, O sacred Virgin.}
\latin{\Rbar Da mihi virt\'utem contra hostes tuos.}
\vern{Give me strength against thine enemies.}
\latin{Or\'emus}
\vern{Let us pray.}
\latin{Conc\'ede, mis\'ericors Deus, fragilit\'ati nostr\ae \ pr\ae \-s\'\i \-dium : \dag \ 
ut qui sanct\ae \ Dei Genitr\'\i cis mem\'oriam \'agimus, * 
intercessi\'onis ejus aux\'\i lio, a nostris iniqui\-t\'a\-tibus resurg\'amus.
Per e\'umdem Christum D\'ominum nostrum.
\Rbar Amen.}
\vern{We beseech thee, O Lord, mercifully to assist our infirmity: that like as we do now commemorate Blessed Mary Ever-Virgin, Mother of God; so by the help of her intercession we may die to our former sins and rise again to newness of life. Through the same Christ our Lord. Amen.}
\latin{\Vbar Divínum auxílium máneat semper nobíscum. \Rbar Amen.}
\vern{May the divine assistance remain always with us. Amen.}
\stopParallel

%\startParallel
%\latin{\Vbar Divínum auxílium máneat semper nobíscum.}
%\vern{May the divine assistance remain always with us.}
%\latin{\Rbar Amen.}
%%\vern{Amen.}
%\stopParallel


\section{Paschaltide}
\label{reginacaeli}

\rubrics{From Easter Sunday to the Saturday after Pentecost.}

\bibleref{Solemn tone:}{}

\greannotation{Ant.}
\greannotation{6.}
\gregorioscore{MarianAntiphons/reginacaeli_solemn}

\bibleref{Simple tone:}{}

\greannotation{6.}
\gregorioscore{MarianAntiphons/reginacaeli}

\begin{verse}
\translationfont{
Queen of heaven, rejoice, alleluia.\\
For He whom thou didst merit to bear, alleluia,\\
has risen as he said, alleluia.\\
Pray to God for us, alleluia.\\
Rejoice and be glad, O Virgin Mary, alleluia.\\
For the Lord has truly risen, alleluia.

}
\end{verse}

\startParallel
\latin{\Vbar Gaude et l\ae t\'are Virgo Mar\'\i a, allel\'uia.}
\vern{Rejoice and be glad O Virgin Mary, alleluia.}
\latin{\Rbar Quia surr\'exit D\'ominus vere, allel\'uia.}
\vern{For the Lord has risen indeed, alleluia.}
\latin{Or\'emus.}
\vern{Let us pray.}
\latin{Deus, qui per resurrecti\'onem F\'\i lii tui D\'omini
 nostri Jesu Christi mundum l\ae tific\'are dignatus es : \dag \ 
pr\ae sta, qu\ae sumus; ut per ejus Genitr\'\i cem 
V\'\i rginem Mar\'\i am, * perp\'etu\ae \ cap\'\i amus
 g\'audia vit\ae . Per e\'umdem Christum D\'ominum nostrum. \Rbar Amen.}
\vern{O God, who through the resurrection of Thy Son our Lord Jesus Christ
didst vouchsafe to give joy to the world:
grant, we beseech thee, that through His Mother, the Virgin Mary,
we may obtain the joys of everlasting life.
Through Christ our Lord. Amen.}
\latin{\Vbar Divínum auxílium máneat semper nobíscum. \Rbar Amen.}
\vern{May the divine assistance remain always with us. Amen.}
\stopParallel

%\startParallel
%\latin{\Vbar Divínum auxílium máneat semper nobíscum.}
%\vern{May the divine assistance remain always with us.}
%\latin{\Rbar Amen.}
%\vern{Amen.}
%\stopParallel


\section{During the year}
\label{salveregina}

\rubrics{From Trinity Sunday to the Saturday before the first Sunday of Advent.}

\bibleref{Solemn tone:}{}

\greannotation{Ant.}
\greannotation{1.}
\gregorioscore{MarianAntiphons/salveregina_solemn}

\bibleref{Simple tone:}{}

\greannotation{5.}
\gregorioscore{MarianAntiphons/salveregina}

\begin{verse}
\translationfont{
Hail, holy Queen, Mother of mercy,\\
hail, our life, our sweetness, and our hope.\\
To thee do we cry, the children of Eve;\\
to thee do we send up our sighs,\\
mourning and weeping in this valley of tears.\\
Turn, then, most gracious advocate,\\
thine eyes of mercy toward us;\\
and after this our exile, show unto us\\
the blessed fruit of thy womb, Jesus:\\
O clement, O loving, O sweet Virgin Mary.

}
\end{verse}

\startParallel
\latin{\Vbar Ora pro nobis sancta Dei Génitrix.}
\vern{Pray for us, O holy Mother of God.}
\latin{\Rbar Ut digni efficiámur promissiónibus Christi.}
\vern{That we may be made worthy of the promises of Christ.}
\latin{Or\'emus.}
\vern{Let us pray.}
\latin{Omn\'\i potens sempit\'erne Deus, qui glori\'osae V\'\i rginis Matris Mar\'\i {\ae}
corpus et \'animam, ut dignum F\'\i lii tui habit\'aculum \'effici merer\'etur,
Sp\'\i ritu Sancto cooper\'ante pr{\ae}par\'asti : \dag \  da, ut cujus 
commemorati\'one l{\ae}t\'amur, * ejus pia intercessi\'one ab
inst\'antibus malis et a morte perp\'etua liber\'emur.
Per e\'umdem Christum D\'ominum nostrum. \Rbar Amen.}
\vern{Almighty, everlasting God, who by the co-operation of the Holy Spirit didst prepare the body and soul of the glorious Virgin-Mother Mary to become a dwelling-place meet for thy Son: grant that as we rejoice in her commemoration; so by her fervent intercession we may be delivered from present evils and from everlasting death. Through the same Christ our Lord. Amen.}
\latin{\Vbar Divínum auxílium máneat semper nobíscum. \Rbar Amen.}
\vern{May the divine assistance remain always with us. Amen.}
\stopParallel


%\startParallel
%\latin{\Vbar Divínum auxílium máneat semper nobíscum.}
%\vern{May the divine assistance remain always with us.}
%\latin{\Rbar Amen.}
%\vern{Amen.}
%\stopParallel


\appendix


\chapter{Sacred Triduum}

\rubrics{On Maundy Thursday, Compline should be celebrated after the altars have been stripped, and on Good Friday at some time after the Solemn Liturgy.  No candles should be lit, and there should be no singing.  On Holy Saturday, Compline is celebrated only by those who have not attended the Vigil.}

\rubrics{Whoever presides should give a sign for the examination of conscience to begin, if this be customary, and the Hour begins simply with the Confiteor, Misereatur and Indulgentiam on a monotone.  The first psalm is simply commenced (Cum invocarem p.~\pageref{sunday}). There are no antiphons, and the Gloria Patri is omitted.  The voice is dropped a whole tone on the last syllable of each verse of the psalms.}

\rubrics{After the three psalms the Canticle Nunc dimittis is recited in the same manner p.~\pageref{canticle}, no antiphons, voice lowered on last syllable.}

\rubrics{On Maundy Thursday, all then kneel and recite together the following: }

\startParallel
\latin{Christus factus est pro nobis obédiens usque ad mortem.}
\vern{Christ was made for us obedient unto death.}
\stopParallel

\rubrics{On Good Friday is said in place of this:}

\startParallel
\latin{Christus factus est pro nobis obédiens usque ad mortem, mortem autem crucis.}
\vern{Christ was made for us obedient unto death, death on the cross.}
\stopParallel

\rubrics{All say the Pater Noster silently}

\rubrics{The Collect is said without Oremus %p.~\pageref{collect} 
again dropping the voice on the last syllable as in the psalms.}

\startParallel
\latin{Vísita, quæsumus, Dómine, habitatiónem istam, et omnes insídias inimíci ab ea lónge repélle: \dag\ Ángeli tui sancti hábitent in ea, qui nos in pace custódiant; * et benedíctio tua sit super nos semper.}
\vern{Visit we beseech thee, O Lord, this dwelling, and drive far from it the snares of the enemy. Let thy holy angels dwell herein to preserve us in peace and let thy blessing be always upon us.}
\stopParallel

\rubrics{The conclusion \emph{Per Dóminum.} is said silently on Thursday and Friday but aloud on Saturday.}

\rubrics{All then rise and leave in silence.}

\chapter{Easter Octave}
\label{EasterOctave}

\rubrics{Begin at the beginning p.~\pageref{beginning} through to the Deus in adjutorium.}

\rubrics{Skip the antiphon and go straight into the Psalms for Sunday: Cum invocárem p.~\pageref{sunday}}

\rubrics{Finish with this quadruple Alleluia:}

\gresetinitiallines1
\greannotation{Ant.}
\greannotation{8.}
\gregorioscore{Psalms/octavealleluia}

\rubrics{Then straight into the Canticle without antiphon:}

\gregorioscore{Canticle/easteroctavenunc}

%\rubrics{Finish with this antiphon:}

\greannotation{Ant.}
\greannotation{2.}
\gregorioscore{Canticle/an--haec_dies--solesmes}

\rubrics{Then the Collect as usual, with all that follows.}

\startParallel
\latin{\Vbar Dómine, exaudi oratiónem meam.}
\vern{O Lord, hear my prayer.}
\latin{\Rbar Et clamor meus ad te véniat.}
\vern{And let my cry come unto Thee.}
\latin{Orémus. Vísita, quæsumus, Dómine, habitatiónem istam, et omnes insídias inimíci ab ea lónge repélle: \dag\ Ángeli tui sancti hábitent in ea, qui nos in pace custódiant; * et benedíctio tua sit super nos semper.
Per Dóminum nostrum Jesum Christum, Filium tuum: \dag\ qui tecum vivit et regnat in unitáte Spíritus Sancti Deus, * per ómnia saécula sæculórum. \Rbar Amen.}
\vern{Let us pray. Visit we beseech thee, O Lord, this dwelling, and drive far from it the snares of the enemy. Let thy holy angels dwell herein to preserve us in peace and let thy blessing be always upon us.
Through Jesus Christ, thy Son our Lord, Who liveth and reigneth with thee, in the unity of the Holy Ghost, ever one God, world without end. Amen.}
\latin{\Vbar Dómine, exaudi oratiónem meam.}
\vern{O Lord, hear my prayer.}
\latin{\Rbar Et clamor meus ad te véniat.}
\vern{And let my cry come unto Thee.}
\latin{\Vbar Benedicámus Dómino. \Rbar Deo grátias.}
\vern{Let us bless the Lord. Thanks be to God.}
\latin{Benedícat et custódiat nos omnípotens et miséricors Dóminus, \maltese Pater, et Fílius, et Spíritus Sanctus. \Rbar Amen.}
\vern{The almighty and merciful Lord the Father, the Son and the Holy Spirit bless us and keep us. Amen.}
\stopParallel

\rubrics{Then Regina Caeli, p.~\pageref{reginacaeli}.}

\chapter{Paschal psalmody}

\rubrics{Most of the psalms already use mode 8G, except Wednesday, Friday and Saturday.  The following psalms have the same words, but all set to 8G.}

\section{Wednesday in Paschal Time}
\label{wed-pt}

\greannotation{P. T.}
\greannotation{Ant.}
\gregorioscore{Psalms/alleluiapaschalinit}


\psalm{33. I}

\gresetinitiallines0
\gregorioscore{Psalms/wednesdayeasterpsalm1init}

\setcounter{versecount}{1}
\startParallel
\input Psalms/wednesdayeasterpsalm1
\stopParallel


\psalm{33. II}

\gregorioscore{Psalms/wednesdayeasterpsalm2init}

\setcounter{versecount}{1}
\startParallel
\input Psalms/wednesdayeasterpsalm2
\stopParallel


\psalm{60}

\gregorioscore{Psalms/wednesdayeasterpsalm3init}

\setcounter{versecount}{1}
\startParallel
\input Psalms/wednesdayeasterpsalm3
\stopParallel



\greannotation{P. T.}
\greannotation{Ant.}
\gregorioscore{Psalms/alleluiapaschal}

\section{Friday in Paschal Time.}
\label{fri-pt}

\gresetinitiallines1
\greannotation{P. T.}
\greannotation{Ant.}
\gregorioscore{Psalms/alleluiapaschalinit}

\psalm{76. I}

\gresetinitiallines0
\gregorioscore{Psalms/fridayeasterps1v1}

\setcounter{versecount}{1}
\startParallel
\input Psalms/fridayeasterps1vv
\stopParallel



\psalm{76. II}

\gregorioscore{Psalms/fridayeasterps2v1}

\setcounter{versecount}{1}
\startParallel
\input Psalms/fridayeasterps2vv
\stopParallel


\psalm{85}

\gregorioscore{Psalms/fridayeasterps3v1}

\setcounter{versecount}{1}
\startParallel
\input Psalms/fridayeasterps3vv
\stopParallel



\gresetinitiallines1
\greannotation{P. T.}
\greannotation{Ant.}
\gregorioscore{Psalms/alleluiapaschal}

\section{Saturday in Paschal Time.}
\label{sat-pt}

\greannotation{P. T.}
\greannotation{Ant.}
\gregorioscore{Psalms/alleluiapaschalinit}

\psalm{87}

\gresetinitiallines0
\gregorioscore{Psalms/saturdayeasterpsalminit1}

\setcounter{versecount}{1}
\startParallel
\input Psalms/saturdayeasterpsalm1
\stopParallel


\psalm{102. I}

\gregorioscore{Psalms/saturdayeasterpsalminit2}

\setcounter{versecount}{1}
\startParallel
\input Psalms/saturdayeasterpsalm2
\stopParallel


\psalm{102. II}

\gregorioscore{Psalms/saturdayeasterpsalminit3}

\setcounter{versecount}{1}
\startParallel
\input Psalms/saturdayeasterpsalm3
\stopParallel


\gresetinitiallines1
\greannotation{P. T.}
\greannotation{Ant.}
\gregorioscore{Psalms/alleluiapaschal}

\setcounter{versecount}{0}




\chapter{All Souls' Day}

%Start with the Confiteor, then these psalms without antiphon
\rubrics{The usual beginning of Compline is omitted today, and the Hour begins from the \emph{Confiteor}, \emph{Misereatur} and \emph{Indulgentiam}.  The following psalms are then recited directly without an antiphon.  \emph{Gloria Patri} is not said, but is replaced with \emph{Requiem aeternam} as shown.  Either of two tones may be used.  Here I have only printed the second as in the Liber Usualis.}

\psalm{122}

\gresetinitiallines0
\gregorioscore{Psalms/allsouls1v1}

\setcounter{versecount}{1}
\startParallel
\input Psalms/allsouls1vv
\stopParallel


\psalm{141}

\gregorioscore{Psalms/allsouls2v1}

\setcounter{versecount}{1}
\startParallel
\input Psalms/allsouls2vv
\stopParallel


\psalm{142}

\gregorioscore{Psalms/allsouls3v1}

\setcounter{versecount}{1}
\startParallel
\input Psalms/allsouls3vv
\stopParallel

\section*{Canticle}

\gregorioscore{Canticle/allsoulsv1}

%\setcounter{versecount}{1}
%\startParallel
%\input Canticle/allsoulsvv
%\stopParallel



\setcounter{versecount}{0}



\rubrics{All kneel to recite the \emph{Pater noster} silently, concluding with:}

\startParallel
\latin{\Vbar Et ne nos indúcas in tentatiónem.}
\vern{And lead us not into temptation.}
\latin{\Rbar Sed líbera nos a malo.}
\vern{But deliver us from evil.}
\latin{\Vbar A porta ínferi.}
\vern{From the gate of hell.}
\latin{\Rbar Erue, Dómine, ánimas eórum.}
\vern{Deliver, O Lord, their souls.}
\latin{\Vbar Requiéscant in pace.}
\vern{May they rest in peace.}
\latin{\Rbar Amen.}
\vern{Amen.}
\latin{\Vbar Dómine, exaudi oratiónem meam.}
\vern{O Lord, hear my prayer.}
\latin{\Rbar Et clamor meus ad te véniat.}
\vern{And let my cry come unto Thee.}\latin{Oremus.}
\vern{Let us pray.}
\latin{Propitiáre, quaésumus, Dómine, animábus ómnium famulórum famularúmque tuárum, pro quibus majestátem tuam supplíciter exorámus: \dag\ ut, per haec piae deprecatiónis offícia, * perveníre mereántur ad réquiem sempitérnam. Per Dóminum nostrum Jesum Christum Fílium tuum, \dag\ qui tecum vivit et regnat in unitáte Spíritus Sancti Deus, * per ómnia saécula saeculórum.} 
\vern{In merciful forgiveness, O Lord, look down upon the souls of all thy servants and handmaids, for whom we humbly entreat thy majesty: that, by our prayers of pious supplication, they may be counted worthy to enter thine everlasting rest. Through our Lord Jesus Christ, thy Son, who livest and reignest with Thee in the unity of the Holy Spirit, one God for ever and ever.}
\latin{\Rbar Amen.}
\vern{Amen.}
\latin{\Vbar Réquiem ætérnam dona eis Dómine.}
\vern{Eternal rest grant unto them, O Lord.}
\latin{\Rbar Et lux perpétua lúceat eis.}
\vern{And may perpetual light shine upon them.}
\latin{\Vbar Requiéscant in pace.}
\vern{May they rest in peace.}
\latin{\Rbar Amen.}
\vern{Amen.}
\stopParallel

\rubrics{Thus ends Compline, nothing further is added.}

%page 1742 Liber Usualis

\end{document}
